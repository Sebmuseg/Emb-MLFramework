% Formatierungseinstellungen für das Dokument

% Seitenränder einstellen
\usepackage{geometry}
\geometry{
    a4paper,
    left=3cm, 
    right=4cm, 
    top=2cm,
    bottom=2cm
}
\usepackage{graphicx}
\usepackage{geometry}
\usepackage{setspace}
\usepackage{titlesec}
\usepackage{fancyhdr}
\usepackage{tocloft}
% Zeilenabstand
\usepackage{setspace}
\onehalfspacing  

% Schriften einstellen
\renewcommand{\baselinestretch}{1.5}
\usepackage{mathptmx} % Times New Roman-like font for text
\usepackage[scaled=.90]{helvet} % Helvetica (Arial-like) for captions and tables
\renewcommand{\familydefault}{\rmdefault}

% Kopf- und Fußzeilen anpassen
\usepackage{fancyhdr}
\pagestyle{fancy}
\fancyhf{}
\fancyhead[L]{\leftmark}  % Linke Kopfzeile: Kapitelname
\fancyfoot[C]{\thepage}   % Zentrierte Fußzeile: Seitenzahl

% Abstand zwischen Absätzen
\setlength{\parskip}{1em}

% Keine Einrückung der ersten Zeile eines neuen Absatzes
\setlength{\parindent}{0pt}

% Hyperlink-Farben (z.B. für Inhaltsverzeichnis und Links)
\usepackage{hyperref}
\hypersetup{
    colorlinks=true,
    linkcolor=black,      % Schwarz für Links auf den Inhalt der Arbeit
    citecolor=darkgray,   % Dunkelgrau für Zitate
    filecolor=gray,       % Mittelgrau für Dateilinks
    urlcolor=teal         % Teal für URLs, sanft aber klar erkennbar
}

\usepackage[ngerman]{babel}  % Correctly use ngerman with babel
\usepackage[autostyle=true]{csquotes}  % Load csquotes without language options

\usepackage{rotating}



