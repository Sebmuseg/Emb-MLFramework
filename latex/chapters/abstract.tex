\begin{abstract}
    In der modernen Industrie 4.0 spielen Embedded Systems und Edge-Computing eine zunehmend entscheidende Rolle in der Automatisierung und Optimierung von Produktionsprozessen. Diese Technologien ermöglichen eine datengestützte und lokale Verarbeitung, die schnelle und intelligente Entscheidungen in Echtzeit erlaubt. Der Einsatz von Machine Learning (ML) in diesen Umgebungen bietet vielversprechende Potenziale, um Effizienzsteigerungen und Kostensenkungen in der industriellen Fertigung zu erzielen. Gleichzeitig stellt die Implementierung von ML auf Embedded Systems eine enorme Herausforderung dar, da diese Systeme oft durch begrenzte Rechenleistung, Speicher und Latenzzeiten eingeschränkt sind.
    
    Diese Arbeit entwickelt ein Framework, das speziell für die Optimierung von ML-Modellen auf ressourcenbeschränkten Embedded Systems ausgelegt ist. Der Fokus liegt dabei auf der Reduzierung der Modellgröße, der Sicherstellung der Echtzeitfähigkeit sowie der Integration in bestehende Steuerungssysteme wie speicherprogrammierbare Steuerungen (SPS) und Industrie-PCs (IPC). Um dieses Ziel zu erreichen, wurde eine strukturierte Literaturrecherche durchgeführt und eine umfassende Anforderungsanalyse in Zusammenarbeit mit Industriepartnern vorgenommen.
    
    Das entwickelte Framework nutzt Optimierungstechniken wie Quantisierung und Model Pruning, um die Effizienz der ML-Modelle zu maximieren, ohne die Genauigkeit zu beeinträchtigen. Es bietet eine flexible Schnittstelle für das Deployment und die Aktualisierung von Modellen und unterstützt ein umfassendes Logging- und Überwachungssystem, um die langfristige Wartung und Fehleranalyse zu erleichtern.
    
    Die Ergebnisse zeigen, dass das Framework in der Lage ist, ML-Modelle effizient auf Embedded Systems auszuführen und die Anforderungen der Industrie 4.0 hinsichtlich Echtzeitfähigkeit und Ressourcenschonung zu erfüllen. Dieses Framework bietet somit eine zukunftssichere Lösung für den Einsatz von Machine Learning in der industriellen Fertigung und ebnet den Weg für weitere Entwicklungen im Bereich Embedded Machine Learning.
    \end{abstract}