\chapter{Einleitung}
\label{chap:einleitung}


In der heutigen Industrie 4.0 spielen Embedded Systems und Edge-Computing eine zentrale Rolle in der Automatisierung und Optimierung von Produktionsprozessen.
Diese Technologien ermöglichen es, Maschinen und Produktionssysteme mit Intelligenz auszustatten, indem sie Daten direkt an der Quelle verarbeiten und schnelle,
datengesteuerte Entscheidungen treffen. Besonders im industriellen Umfeld, wo Ressourcen wie Speicher und Rechenleistung oft begrenzt sind, stellt der Einsatz von \ML 
auf Embedded Systemen eine vielversprechende, aber zugleich herausfordernde Möglichkeit dar.

Die Relevanz von \ML in der Industrie ist unbestritten. 
Von der vorausschauenden Wartung bis hin zur Qualitätskontrolle – \ML bietet zahlreiche Möglichkeiten zur Effizienzsteigerung und Kostensenkung. 
Allerdings sind viele bestehende Machine-Learning-Modelle nicht für den Einsatz auf Embedded Systemen optimiert. 
Die besonderen Herausforderungen, die sich durch eingeschränkte Ressourcen und strenge Echtzeitanforderungen ergeben, machen es notwendig, 
spezialisierte Ansätze für das Modell-Deployment auf solchen Systemen zu entwickeln.

Das Hauptziel dieser Arbeit ist es, ein optimiertes Modell-Deployment auf Embedded Systemen zu realisieren, das den spezifischen Bedingungen in der industriellen Umgebung gerecht wird. 
Dabei werden sowohl die technischen Herausforderungen als auch die praktischen Implikationen des Einsatzes von \ML auf speicher- und rechenleistungseingeschränkten Systemen untersucht.

Um dieses Ziel zu erreichen, wird in dieser Arbeit eine Kombination aus modellbasierten und experimentellen Ansätzen verwendet. 
Zunächst werden die theoretischen Grundlagen von Embedded Systems und Machine Learning in der Industrie 4.0 untersucht. 
Anschließend wird ein Framework entwickelt, das speziell auf die Anforderungen von Embedded Systemen abgestimmt ist. 
Dieses Framework wird in einer industriellen Umgebung getestet, um seine Leistungsfähigkeit und Effizienz zu validieren.

Die Arbeit ist wie folgt strukturiert: In Kapitel \ref{chap:theoretische_hintergrund} werden die theoretischen Grundlagen und der Stand der Technik im Bereich Embedded Systems und Machine Learning in der Industrie 4.0 erörtert. 
Kapitel \ref{chap:methodik} beschreibt die Methodik, die bei der Entwicklung des Frameworks angewendet wurde. 
In Kapitel \ref{chap:entwicklung_framework} wird das entwickelte Framework im Detail vorgestellt, gefolgt von der Implementierung und Optimierung in Kapitel \ref{chap:implementierung_optimierung}. 
Die Evaluation des Frameworks wird in Kapitel \ref{chap:evaluation} behandelt, bevor in Kapitel \ref{chap:diskussion} die Ergebnisse diskutiert werden. 
Abschließend fasst Kapitel \ref{chap:fazit} die Arbeit zusammen und gibt einen Ausblick auf zukünftige Entwicklungen.

