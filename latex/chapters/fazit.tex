\chapter{Fazit}
\label{chap:fazit}

Diese Arbeit befasst sich mit der Entwicklung eines Frameworks zur effizienten Ausführung und Optimierung von Machine-Learning-Modellen 
auf ressourcenbeschränkten Embedded-Systemen. Ziel war die Schaffung eines flexiblen, anpassungsfähigen und ressourceneffizienten Frameworks, 
das die Anforderungen industrieller Anwendungen erfüllt, insbesondere hinsichtlich Echtzeitleistung und Speicheroptimierung.

Die zentralen Beiträge dieser Arbeit umfassen die Entwicklung eines modularen Frameworks, das eine flexible Integration von ML-Modellen 
auf verschiedenen Embedded- und Edge-Geräten ermöglicht. Die modulare Architektur erlaubt es, unterschiedliche Modelle effizient zu 
optimieren und an die spezifischen Anforderungen von Ressourcenbeschränkungen anzupassen. Durch die Unterstützung von 
Optimierungstechniken wie Quantisierung, Pruning und Modellkompression wird die Ausführungszeit reduziert und der Speicherverbrauch 
deutlich verringert, ohne die Modellgenauigkeit erheblich zu beeinträchtigen. Das Framework erfüllt die Echtzeitanforderungen, 
die in industriellen Anwendungen unabdingbar sind, und bietet durch gezielte Priorisierung und effiziente Modellverarbeitung eine hohe Vorhersagegeschwindigkeit.

Ein wesentlicher Mehrwert des Frameworks liegt in seiner breiten Unterstützung für diverse Hardwareplattformen, 
einschließlich speicherprogrammierbarer Steuerungen (SPS), Industrie-PCs (IPCs), Mikrocontroller und leistungsstarker Edge-Devices. 
Diese breite Plattformunterstützung ermöglicht den Einsatz des Frameworks in einer Vielzahl industrieller Umgebungen, 
von stark ressourcenbeschränkten Systemen bis hin zu leistungsfähigen Edge-Geräten.

Trotz der erzielten Fortschritte gibt es gewisse Einschränkungen. Die Optimierungstechniken führen in einigen Fällen zu 
geringfügigen Genauigkeitsverlusten, die in sicherheitskritischen Anwendungen relevant sein könnten. Zudem wurde die Kompatibilität 
des Frameworks mit speziellen Hardwareplattformen wie FPGAs oder neueren spezialisierten Edge-Chips nicht vollständig untersucht. 
Eine weiterführende Integration adaptiver Lernmechanismen, die sich dynamisch an die Systemauslastung oder veränderte Bedingungen anpassen, 
könnte die Flexibilität des Frameworks erhöhen.

Basierend auf diesen Erkenntnissen ergeben sich mehrere Ansätze für zukünftige Arbeiten. Eine vertiefte Entwicklung der Optimierungstechniken, 
insbesondere der Einsatz von Quantization-Aware Training (QAT), könnte Genauigkeitsverluste nach der Quantisierung weiter minimieren. 
Die Erweiterung auf zusätzliche Hardwareplattformen, wie spezialisierte Edge-Prozessoren und FPGAs, könnte die Leistungsfähigkeit und 
Energieeffizienz des Frameworks erhöhen. Darüber hinaus stellt die Implementierung adaptiver Modelle mit dynamischem Task-Management ein 
vielversprechendes Forschungsthema dar. Solche Modelle könnten in Echtzeitsystemen von großem Nutzen sein, besonders bei schwankenden 
Systemressourcen oder Umweltbedingungen.

Zusammenfassend zeigen die in dieser Arbeit vorgestellten Methoden und Ansätze, dass Machine-Learning-Modelle effizient auf Embedded- und 
Edge-Systemen eingesetzt werden können, auch wenn diese nur über begrenzte Ressourcen verfügen. Die Ergebnisse bestätigen, 
dass durch eine flexible und modulare Framework-Architektur sowie gezielte Modelloptimierungstechniken die Hürden für den Einsatz von 
Künstlicher Intelligenz in industriellen Echtzeitanwendungen gesenkt werden können. Das Framework bietet eine solide Grundlage für z
ukünftige Entwicklungen und zeigt Potenzial für die Optimierung und Automatisierung von Produktionsprozessen in unterschiedlichen industriellen Kontexten. 
Weitere Arbeiten werden auf diesen Ergebnissen aufbauen und sowohl neue Optimierungsansätze als auch eine erweiterte Hardwareunterstützung integrieren, 
um die Anwendbarkeit und Leistungsfähigkeit des Frameworks weiter zu steigern.